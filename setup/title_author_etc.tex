\titulolinhas{}

%% [sub-título]{Título original}
\titulolinhas[]{Compressão de Imagens Usando Redes Neurais Artificiais}

\maketitle

%% No caso de haver outros autores (máximo 3), crie mais um "\autores"
\autores{Marcelo Guedes França}

%% {Grau desejado em detalhes}{Tipo de monografia}{Departamento}{Grau desejado}{Sigla do departamento}{Programa do aluno ou departamento novamente]}{Nome da faculdade}{Sigla da Faculdade}
\grau{Bacharel}{Trabalho de Conclusão de Curso}{Engenharia El\'{e}trica}{ENE}{Engenharia Elétrica}{Faculdade de Tecnologia}{FT}{Electric Engineering}

\datainfo{Novembro}{27}{2019}
\datainfoAnotherLang{November}{27th}

%% {Traduzido para outro língua}{Título original}
\titulolinguas{}{Image Compression using Artificial Neural Networks}

%% Senão houver co-orientador, deixe o campo [] vazio.
\orientador[]{Prof. Dr. Eduardo Peixoto Fernandes da Silva}

%% Para adicionar mais membros da banca: Após o último membro, crie uma nova linha (enter) e vá na aba superior esquerda (abaixo de “file”) e selecione “membro da banca”. Para deletar membros, apenas delete o item correspondente
\membrodabanca[Orientador]{Prof. Dr. Eduardo Peixoto Fernandes da Silva \textendash{} ENE/Universidade de Brasília}

\membrodabanca[Membro Interno]{Prof. Dr. Dr. Edson Mintsu Hung \textendash{} ENE/Universidade de Brasília}

\membrodabanca[Membro Interno]{Prof. Dr.  L. Diogo \textendash{} CIC/Universidade de Brasília}

\catalogonome[França, M.]{França, Marcelo}

%% {Publicação N#}{Palavra chave #1}{chave #2}{chave #3}{chave #4}
\catalogoinfo{PPGEA.TD-001/11}{Compressão de Imagens}{Redes Neurais Artificiais}{Autocodificadores}{Redes Recorrentes}{Image Compression}{Artificial Neural Networks} {Autoencoder}{Recurrent Networks}

%Esparsidade
%Alocação de Bits
%Codificação de Entropi