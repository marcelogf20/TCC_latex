% Comment this as you don`t use it
%-----------------------------------------------
% Use os simbolos, acronimos e notacoes com as funcoes a seguir
% To use this acronyms list, check the file glossaries-list/symbos.tex.
%%---------------------------------------

% Example for Symbols (USE A TAG "symb:" for symbols):
% \newglossaryentry{symb:pi}
% {
%     type=symbolslist,
%     parent=greekletters,
%     sort=a,
%     symbol={\ensuremath{\pi}}
%     name=pi,
%     plural={pi},% the plural hardcoded
%     description={Ratio of circumference of circle to its diameter}
% }

% In the text:
%     use \gls{label}, for an first time extended name
% In the text , for plural:
%     \glspl{duck}
% In the text , for Caps and/or plural:
%     \Gls{duck} => Duck
%     \Glspl{duck} => Ducks
% \glsdesc{<your key>} for description in text
% \Glsdesc{<your key>} for description in text 
% \GLSdesc{<your key>} for description in text
%
%
% To just add a Symbol without using it, do:
% \glsaddall
% \glsadd{label}

%******** Groups ***********
\newglossaryentry{greekletters}{name={Greek symbols},unit={},description={\nopostdesc\newline},sort=a}
\newglossaryentry{latinletters}{name={Latin symbols},unit={},description={\nopostdesc\newline},sort=b}
\newglossaryentry{nodim}{name={Dimensionless symbols},unit={},description={\nopostdesc\newline},sort=c}
\newglossaryentry{subs}{name={Subscript symbols},unit={},description={\nopostdesc\newline},sort=d}
\newglossaryentry{supers}{name={Superscript symbols},unit={},description={\nopostdesc\newline},sort=e}

%******** Latins ***********
\newglossaryentry{symb:E}
{
	type=symbolslist,
	parent=latinletters,
	name={E},
	unit={\si{\joule}},
	description={total energy consumed}
}

\newglossaryentry{symb:P}
{
	type=symbolslist,
	parent=latinletters,
	name={P},
	unit={\si{\watt}},
	description={average power consumption}
}


%******** Greeks ***********
\newglossaryentry{symb:pi}
{
	type=symbolslist,
	parent=greekletters,
	sort=a,
	unit={},
	name={\ensuremath{\pi}},
	description={ratio of circumference of circle to its diameter}
}

%******** No dimensions ***********
\newglossaryentry{symb:dummy}
{
	type=symbolslist,
	parent=nodim,
	name={Dummy},
	unit={},
	description={a dummy counter}
}

%******** sub entries ***********
\newglossaryentry{symb:idle}
{
	type=symbolslist,
	parent=subs,
	name={\ensuremath{Idle}},
	unit={},
	description={the idle state of an energy}
}


%******** super entries ***********
\newglossaryentry{symb:point}
{
	type=symbolslist,
	parent=supers,
	name={\ensuremath{\dot{y}}},
	unit={},
	description={the first derivative for a function $y=f(x)$}
}


% To use this acronyms list, check the file glossaries-list/acronyms.tex.
% Example for Glossary:
% \newglossaryentry{latex}
% {
%     name=latex,
%     description={Is a mark up language specially suited 
%     for scientific documents}
% }
%
% To include acronyms into the list of symbols, do as follow:
% * add [parent=acronymslists] after every "\newacronym";
% * Comment the line "\listadeacronimos" in file "thesis.tex";
% * Remove the comment the following line:
%\newglossaryentry{acronymslists}{name={Acronyms list},description={\glspar},sort=c}


\newglossaryentry{maths}
{
    name=mathematic,
    description={Mathematics is what mathematicians do}
}

% Example for Acronyms:
% \newacronym{label}{Acronym}{Full name}
\newacronym{rna}{RNA}{Rede Neural Artificial (\textit{Artificial Neural Network})}
\newacronym{relu}{ReLU}{Unidade Linear Retificadora (\textit{Rectified Linear Unit})}
%\newacronym{tanh}{}{Tangente Hiperbólica\textit(Hyperbolic tangent})}
\newacronym{rnc}{RNC}{Rede Neural Convolucional (\textit{Convolutional Neural Network})}
\newacronym{conv2d}{Conv2D}{Convolução Bidimensional}

\newacronym{conv2dlstm}{Conv2DLSTM}{Rede de Mémória de Longo Prazo Convolucional Bidimensional}

\newacronym{admm}{ADMM}{Método de Direção Alternativa dos Multiplicadores (\textit{alternating direction method of multipliers})}



\newacronym{conv2drnr}{Conv2DRNR}{Rede Neural Recorrente Convolucional Bidimensional}
\newacronym{pe}{PE}{Profunidade para Espaço (\textit{Depth-to-Space}) }
\newacronym{gru}{GRU}{Unidades Recorrentes Fechadas (\textit{Gated Recurrent Units})}

\newacronym{rgb}{RGB}{Vermelho, Verde, Azul (\textit{Red, Green, Blue})}
\newacronym{rnr}{RNR}{Rede Neural Recorrente (\textit{Recurrent Neural Network})}

\newacronym{mlp}{MLP}{Perceptron Multicamadas (\textit{Multilayer Perceptron})}
\newacronym{lstm}{LSTM}{Memória de Longo Prazo (\textit{Long Short-Term Memory})}
\newacronym{fc-lstm}{FC-LSTM}{LSTM Totalmente Conectada (\textit{Fully-Conected LSTM})}
\newacronym{ac}{AC}{Autocodificador (\textit{autoencoder})}
\newacronym{vae}{VAE}{autocodificador variacional (\textit{variational autocoders})}
\newacronym{codec}{CODEC}{Codificador/Decodificador (\textit{Encoder/Decoder})}
\newacronym{jpeg}{JPEG}{Grupo de Especialistas em Fotografia (\textit{Joint Photographic Experts Group})}
\newacronym{ccitt}{CCITT}{ \textit{Consultative Committee for International Telephony and Telegraphy}}
\newacronym{itu}{ITU-T}{International Telecommunication Union}
\newacronym{iso}{ISO}{International Organization for Standardisation}
\newacronym{ycbcr}{YCbCr}{Luminância, Crominância Azul, Crominância Vermelho (\textit{Luminance, Blue Chrominance, Red Chrominance})}
\newacronym{psnr}{PSNR}{Relação Pico-Sinal-Ruído (\textit{Peak Signal to Noise Ratio})}
\newacronym{ssim}{SSIM}{Índice de Similaridade Estrutural (\textit{Structural Similarity Index})}
\newacronym{msssim}{MS-SSIM}{Índice de Similaridade Estrutural em Multi-Escalas (\textit{Multi-scale Structural Similarity Index})}
\newacronym{wavelet}{DWT}{Discrete Wavelet Transform}
\newacronym{dct}{DCT}{Transformação Discreta de Cosseno (\textit{Discrete Cosine Transform})}

\newacronym{idct}{IDCT}{Transformada Discreta de Cosseno Inversa (\textit{Inverse Discrete Cosine Transform})}

\newacronym{rle}{RLE}{Codificação de Comprimento de Execução (\textit{Run-Length Encoding})}

\newacronym{bpg}{BPG}{Better Portable Graphics}
\newacronym{gan}{GAN}{ Redes Adversárias Generativas (\textit{Generative adversarial network})}
\newacronym{mae}{MAE}{Erro Absoluto Médio (\textit{Mean Absolute Error})}
\newacronym{mse}{MSE}{Erro Médio Quadrático (\textit{Mean Square Error})}
\newacronym{cbr}{CBR}{Taxa de Bits Constante (\textit{Constant Bitrate})}

\newacronym{cart}{CART}{Classification and Regression Trees}

\newacronym{h264}{H.264/AVC}{H.264 Advanced Video Coding}
\newacronym{h265}{H.265/HEVC}{H.265 High Efficiency Video Coding}


\newacronym{ctu}{CTU}{Coding Tree Unit}
\newacronym{ctb}{CTB}{Coding Tree Block}
\newacronym{pu}{PU}{Prediction Unit}

\newacronym{mpeg}{MPEG}{Moving Pictures Experts Group}
\newacronym{ldf}{LDF}{Linear Discriminant Function}
\newacronym{mvvd}{MVVD}{Motion Vector Variance Distance}
\newacronym{mv}{MV}{Motion Vector}
\newacronym{rhi}{RHI}{Residual Homogeneity Indicator}
\newacronym{sar}{SAR}{Sum of Absolute Residual}
\newacronym{qp}{QP}{Quantization Parameter}
\newacronym{gop}{GOP}{Group Of Pictures}
\newacronym{cu}{CU}{Coding Unit}
\newacronym{rd}{RD}{Rate-Distortion}
\newacronym{mi}{IM}{Informação Mútua}
\newacronym{mb}{MB}{Macrobloco}

\newacronym{bdrate}{BD-Rate}{Bj{\o}ntegaard Delta Bitrate}
\newacronym{led}{LED}{Light Emiting Diode}
\newacronym{ctc}{CTC}{Common Test Conditions}
\newacronym{sbtvd}{SBTVD}{Sistema Brasileiro de Televisão Digital}

% In the text:
%     use \gls{label}, for an first time extend name, plus acronyms
%     use \acrshort{label}, for an acronym only
%     use \acrlong{label}, for a full name only
% In the text , for plural:
%     \glspl{duck}
% In the text , for Caps and/or plural:
%     \Gls{duck}
%     \Glspl{duck}




% To use this acronyms list, check the file glossaries-list/notation.tex.
%% Example for Notation (USE A TAG "not:" for notations):
% \newglossaryentry{not:latex}
% {
%     type=notation,
%     name=Latex,
%     text={latex},% This can be used for different Caps in text
%     description={Is a mark up language specially suited 
%     for scientific documents
%     }
% }
%
% In the text:
%     use \gls{label}, for an first time extended name
% In the text , for plural:
%     \glspl{duck}
% In the text , for Caps and/or plural:
%     \Gls{duck} => Duck
%     \Glspl{duck} => Ducks
%
%
% To just add a Symbol without using it, do:
% \glsaddall
% \glsadd{label}

\newglossaryentry{not:vet}
{
	type=notation,
	name=Vet,
	text={vet},
	description={Neste trabalho vetores são representados por letras minúsculas em
		negrito. Matrizes são representadas por letras maiúsculas em negrito.
		Já espaços e conjuntos em geral são representados por letras maiúsculas
		caligráficas.}
}





 

%-----------------------------------------------

%---------------------------------------
% List of symbols definition, do not change it!
%---------------------------------------
\newglossarystyle{symbolsstyle}{%
% 	\setglossarystyle{index}% base this style on the list style
	\renewcommand{\glsgroupskip}{}% make nothing happen between groups
    \setglossarystyle{long3col}% base this style on the list style
    \renewenvironment{theglossary}%
    {\begin{longtable}{>{\raggedright}p{.10\textwidth}p{.75\textwidth}>{\raggedright}p{.20\textwidth}}}%
    {\end{longtable}}%
    
    % If you need the header for this table...
    % \renewcommand*{\glossaryheader}{%  Change the table header
    %   \bfseries Sign & \bfseries Description & \bfseries Unit \\
    %   \hline
    %   \endhead}
    
    \renewcommand*{\glossentry}[2]{%  Change the displayed items
        \mbox{\glstarget{##1}{\glossentryname{##1}}}
        & & \tabularnewline
    }
    \renewcommand{\subglossentry}[3]{%  Change the displayed items
        \glstarget{##2}{\glossentryname{##2}} %
        & \Glossentrydesc{##2}% Description
        & \ifthenelse{\equal{\glsentryunit{##2}}{}}{}{[\glsentryunit{##2}]}  \tabularnewline
    }
}

