\chapter{Introdução}

O século XXI tem sido marcado por grandes avanços tecnológicos que mudaram a forma do ser humano de se comunicar, relacionar, e informar, entreter, etc. Dentre os avanços, destacam-se o advento e popularização da rede mundial de computadores e das redes móveis para celulares. Hoje, elas superaram a distância física para prover comunicações rápidas entre os seres humanos. Essencial também são os avanços dos \textit{smartphones} e computadores que dispõem, cada vez mais, de recursos de \textit{hardware} e \textit{software} para processar grande volume de informação na forma de áudio, imagem, vídeo, etc.

Não é exagero dizer que esse cenário só foi possível com a digitalização e compressão da informação. O avanço na qualidade de mídias como imagem e vídeo é, via de regra, seguido por necessidade maior de dados para representá-las. Como consequência, essas informações  requerem mais espaço em memória para serem armazenadas e maior consumo de largura de banda e energia elétrica quando são transmitidas pela Internet. Todos esses recursos são escassos e dispendiosos. Uma solução viável é a elaboração de algoritmos de compressão cada vez mais eficientes. 

Para imagens, temos observado o surgimento de definições em altas resoluções, como as imagens em \acrshort{hd} ($1280 \times 720$ \textit{pixels}) e chegando até em 10k ($10240 \times 5760$ \textit{pixels}). Aliado a isso, as imagens representam parte significativa no fluxo de informação que trafegam pela Internet. Portanto, faz-se necessário o esforço da comunidade científica, empresas e demais interessados no desenvolvimento de novos \acrshort{codec}s para imagens ou aperfeiçoamento dos existentes. 

%Esses algoritmos foram desenvolvidos para atuam sobre uma categoria específica de informação, como textos, vídeos ou imagens.    

%vantagens na compressão de imagens por efetuar funções extremamentes complexas e não-lineares. O problema de compressão é conhecido pela sua não linearidade. 

As \acrshort{rna}s são conhecidas por ser uma ferramenta para a solução de uma variedade de problemas de classificação e regressão com desempenho, em muitos deles, acima dos métodos tradicionais, mas para o problema de compressão de imagens, as propostas baseadas em \acrshort{rna}s ainda não superam os \acrshort{codec}s convencionais. 
Todavia, os mais recentes trabalhos em compressão de imagens com essa abordagem se mostram com grande potencial para, no futuro, superar o estada da arte \cite{FullResolution2017Toderici,Priming2017Johnston,Autoregressive2018Minnen,Variational2018Balle,zhao1901cae,akbari2019dsslic}.     
Com essa perspectiva, tomamos como base uma arquitetura de autocodificador para compressão de imagens de alta resolução com camadas recorrentes e convolucionais proposto por Toderici et al. \cite{FullResolution2017Toderici} e propomos alterações na tentativa de melhorar o seu desempenho. Considerando essa arquitetura, são objetivos deste trabalho:

 \begin{enumerate}
	\item Analisar a performance da arquitetura em função da entropia da base de dados de treinamento;
	\item Definir uma função de custo correlacionada com a distorção perceptiva entre imagens;
	\item Adicionar uma penalização à função de custo a fim de reduzir a taxa de compressão após uma codificação de entropia;
	\item Propor um método para alocação dinâmica dos bits em uma etapa pós treinamento. 
\end{enumerate}

%Neste trabalho, tomamos como base uma arquitetura de autocodificador para compressão de imagens de alta resolução com camadas recorrentes e convolucionais proposto em \cite{FullResolution2017Toderici} e realizamos algumas análises e alterações na tentativa de melhorar a sua performance. Considerando essa arquitetura, são objetivos deste trabalho:




%As \acrshort{rna}'s possuem vantagens na compressão de imagens por efetuar funções extremamentes complexas e não-lineares. O problema de compressão é conhecido pela sua não linearidade.  Além disso, essas abordagens possuem vantagens na especialização em compressão de imagens com conteúdo específico. O autocodificador é uma arquitetura de \acrshort{rna} amplamente usada na tarefa de compressão de imagens. 

% Ele deve apresentar flexibilidade semelhante aos codecs modernos, atendendo aos seguintes requisitos:

% \begin{enumerate}
%	\item Comprimir imagens de diferentes resoluções.
% 	\item Fixar uma taxa nominal de bits por pixel a ser enviado do codifcador para o decodificador em estágios ou iterações que permite o %controle da taxa de compressão das imagens. 
% 	\item Recuperar a imagem de forma progressiva, o que significa que quanto mais bits forem enviados, mais precisa será a  sua reconstrução. 
%\end{enumerate}










